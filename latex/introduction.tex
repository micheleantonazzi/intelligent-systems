\section{Introduction}\label{header-n3}

In the last few years, advances in technologies and research have had a
great impact on the development of robotics. The robots are employed
every day in a large variety of contexts. They substitute humans in
those activities that can be performed more quickly and precisely. An
example is manufacturing when the production process is automatized
using artificial agents to improve productivity and reduce costs. In
this case, the robots are fixed manipulators with a limited range of
motions, that depends on where it is bolted down. This characteristic
strongly limits the agent's possibilities. Generally, a fixed robot is
programmed to perform a single precise task and it operates in a
controlled environment. This means that the algorithm foresees every
possible situation and often it is coded as a state machine. In
contrast, mobile robots would be able to travel within the environment
in which they operate, applying their talents wherever it is most
effective. Thank mobility, the robotics applications become almost
limitless. Some of them are healthcare, entertainment, and rescue.
Mobile robots are also employed in those tasks that are impossible or
too dangerous for humans, as the exploration of hostile environments: a
building on fire, the seabeds, or the surface of another planet. These
robots can be controlled by humans or can be autonomous. The firsts are
controlled through remote controls while the seconds perceive the
environment and move autonomously according to their task, without human
intervention. The main problem that a mobile robot has to solve is how
to move inside the environment. The first aspect is the \emph{motion
	control}. Each robot has a different locomotion system, specific for the
characteristics of the environment in which it moves. Given its
low-level complexity, the motion actions are performed by a specific
software component. To perform the motion control task is necessary to
use the kinematics: the study of how the robot's mechanical systems
behave. To define the kinematics of a robot, it is necessary to define a
geometrical model (specific for the mechanical characteristics of the
locomotion system) that allows expression of robot motion in a global
reference frame and in the robot's local reference frame. Using this
notation, it is possible to define the robot's kinematics model that
describes the movements and their constraints as a function. Through
kinematics, it is resolved the significant challenge of \emph{position
	estimation}. The next step is the \emph{perception}. An autonomous
system has to acquire knowledge about the environment. This is done by
taking measurements using various sensors and then extracting
information from those measurements. With this information, a mobile
robot can determine its position in the environment. This activity is
called \emph{localization}. The last step for an autonomous mobile agent
is \emph{navigation}. Given partial knowledge about its environment and
a goal position or a series of positions, navigation is the ability of
the robot to act based on its knowledge and sensor values to reach its
goal positions as efficiently and as reliably as possible. There are two
main sub-task of navigation: \emph{path planning} and \emph{obstacle
	avoidance}. The first involves identifying a trajectory that will cause
the robot to reach the goal location when executed. The second consists
of modulating the trajectory of the robot in order to avoid collisions.
Using the techniques explained before, an autonomous mobile robot is
able to robustly navigate inside an environment to perform its tasks.
However, a mobile robot operates in a highly non-deterministic context
and the conventional algorithms often are not suitable or not robust
enough. In the real world, in fact, there are a lot of different tasks
that are too complicated to be modeled by a conventional algorithm. Some
problems indeed may have a wide amount of data difficult to analyze. In
this case, build a specific algorithm means to understand the complex
patterns and the hidden correlations between the data. Instead, other
tasks may be influenced by a lot of external factors that generate a
large quantity of similar but different data. These factors are not easy
to model, especially considered all together, and often they are not a
priori known. This means that an algorithm performs well only in a
controlled environment, that respects specific preconditions. On the
other hand, if it is applied in the real world, the algorithm may
encounter data that it cannot correctly analyze. A particular field of
Computer Science is particularly suitable to solve these situations:
\emph{machine learning} (ML). It represents a family of algorithms that
learn automatically through experience. These algorithms are not
designed for a specific task but they are general purposes so they can
be used to solve each type of task. The principle behind machine
learning is the following: each real phenomenon can be modeled as an
unknown mathematical function which can be approximate by a machine
learning algorithm. In this work, the focus is posed on \emph{deep
	learning} and its application to robotics. Deep learning is based on
artificial neural networks, inspired by the biological neural network
that composed the animal brains. In the following sections are resumed
some papers that apply deep learning to robotics. Each of them specifies
the article title, the name of the journal where it has been published,
and the publication year. For each article is summarized the approach
proposed, the innovation with respect to the literature and the
achievements.